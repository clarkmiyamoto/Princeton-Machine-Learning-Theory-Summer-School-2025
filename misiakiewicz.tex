\section{Overview}
\begin{itemize}
	\item Reductions
	\item Kernel methods
	\item Noisy Gradient Descent
	\item SQ Algorithms
	\item Low degree polynomials (if we have time)
\end{itemize}

\section{Supervised Learning}
Here is the mathematical setup for supervised learning.\\

Consider a dataset $\mathcal D=\{(y_i, x_i)\}_{i \leq n} \sim_{iid} \mathcal P(\mathcal Y \times \mathcal X)$, where $x_i \in \mathcal X$ (called the covariate space) and $y_i \in \mathcal Y$ (called the label space).\\
\\
We have a loss function $\ell : \mathcal Y \times \mathcal Y \to \mathbb R_{\geq 0}$. The goal is to fit a model $\hat f: \mathcal X \to \mathcal Y$, which minimizes the \textbf{population risk}
\begin{align}
	\mathcal R(\hat f, \mathcal D) \equiv \mathbb E_{(y,x)_\sim \mathcal D} [\ell (y, \hat f(x))]
\end{align}
\begin{sidework}
	\emph{Example:} Let $\mathcal X = \mathbb R^d, \mathcal Y = \mathbb R$, and the loss function is the residuals $\ell=  (y - \hat y)^2$. Your population risk can be the mean squared error $\mathcal R = \mathbb E[(y - \hat f)^2]$. 
	We define the \textbf{excess test error}
	\begin{align}
		\mathcal R(\hat f, \mathcal D) & = \mathcal R(\hat f, \mathcal D0 - \inf_{f} \mathcal R(\ell, \mathcal D)\\
		& = \mathbb E[(h(x) - \hat f)^2] , ~~\text{s.t. } h(x) = \mathbb E[y|x]
	\end{align}
\end{sidework}
To start constructing bounds, we must apply a little more setup. In this discipline, the workflow goes as follows:
\begin{enumerate}
	\item Choose a model class
	\item Setup the empirical risk minmization problem
	\item See what is achievable via some gradient descent type algorithm
\end{enumerate}


To start constructing bounds, let's assume our model is within some function class $\hat f \in \mathcal F = \{f(x;\theta) : \theta \in \mathbb R^p\}$. And going forward, we'll notate the population risk as 
\begin{align}
	\mathcal R(\theta) \equiv \mathcal R(f(\cdot; \theta), \mathcal D)	
\end{align}
We also don't have oracle access to the true distribution $\mathcal P(\mathcal Y \times \mathcal X)$, so we will have to use an \textbf{empirical risk}
\begin{align}
	\hat{\mathcal R}_n(\theta) = \frac{1}{n} \sum_{i=1}^n \ell(y_i, f(x_i; \theta))\\
	\hat \theta = \text{argmin}_\theta \{\hat R_n(\theta) + \pi (\theta)\}
\end{align}
where $\pi$ is a regularizer (for example, you can do Ridge which is $\pi(\theta) = ||\theta||_2^2$).
\\
\\
There are three problems which arrises in the supervised learning setup.
\begin{enumerate}
	\item \textbf{Expressivity:} The model class $\mathcal F$ is not expressive enough. That is you can't capture the relationship between $y$ and $x$ well enough. However, most neural networks are expressive enough, so we can ignore this issue.
	\item \textbf{Sample size:} In practice, your empirical risk is performing a MCMC approximation of the population risk. Meaning  $\hat R_n(\theta) \slashed \simeq \mathcal R(\theta)$. This means $\hat f$ will fail to generalize.
	\item \textbf{Runtime:} It's computational hard to minimize $\hat R_n(\theta)$ as it is highly non-convex.
\end{enumerate}
We will focus on the third, as they play nice with lower bounds.


\section{Reductions}
Reductions are finding mapping between problems. If one problem is hard, then the correct map can show related problems are also hard. We'll show how to determine the 3 Node Neural Network is hard via a mapping.
\subsection{Hardness of Training NNs}
\subsection{3 Node Neural Network}
[Judd '87] [Blum \& Rivest '88]. Assuming $\mathsf{P} \neq \mathsf{NP}$. There is no polynomial time algorithm that (unconditonally) minimizes the empirical risk.\\
\\
Assume you have a function $f(x_i; \theta) = \text{sgn}(\langle w_i, x\rangle + b_i)$, where the ERM
\begin{align}
	\min_\theta \frac{1}{n} \sum_i (y_i - f(x_i; \theta))^2
\end{align}
where the dataset is $y_i \in \{\pm \}$ , $x_i \in \{0,1\}^d$.

\emph{Question:} Given a set of $\mathcal O(n)$ data points, does there exists a 3 node NN  that interpolates this data? \emph{Answer [Blum, Rivest]:} Assuming $\mathsf P \neq \mathsf{NP}$, training a 3 node NN is NP-complete.

\begin{sidework}
\emph{Set Splitting Problem:}
	Given a finite set $S$ and a collection of subset $ C = \{C_j : C_j \subseteq S\}$. Does there exists a partition of $S$ in $S_1 \& S_2$ s.t. $C_j \slashed \subseteq S_1$ and $C_j \slashed \subseteq S_2$ for all $C_j \in C$? That answer is no.

To prove this, just have to map $(S, C) \to (y_i, x_i)_{i\leq n}$, and you can use the previous result to finish the proof.
\end{sidework}
Does this mean that machine learning is doomed? In other words, is this a compelling result? Well:
\begin{enumerate}
	\item This is a worst case result
	\item And we've assumed a very particular architecture.
\end{enumerate}
As a rebuttal, in machine learning
\begin{enumerate}
	\item Average case hardness $\neq$ worst case hardness
	\item This previous example was proper, in machine learning we are often doing improper learning.
\end{enumerate}
\begin{definition}
	[(Im)proper Learning] Consider the function class $\mathcal H \subseteq \{h : \mathcal X \to \{\pm 1\}\}$.
	\begin{itemize}
		\item \textbf{Proper Learning} is when $(y_i, x_i) \sim_{iid} \mathcal D$. The algorithm needs to output $\hat h \in \mathcal H$, s.t. $\mathcal R(\hat f , \mathcal D) \leq \epsilon$.
		\item \textbf{Improper Learning} is when $\hat h \notin \mathcal H$ 
	\end{itemize}
	Basically in proper learning, you constrain the class of models to be a subset of what is defined by the data. In improper learning, you can (for example) over parameterized, and then find solutions this way.
\end{definition}
Once we realize there's a difference between proper / improper learning, we are saved! We can show that hardness of proper learning does not imply hardness of improper learning.

\begin{sidework}
	\emph{Example [Learning Degenerative Normal Forms]:} Let $x = (x_1, x_2, ...) \in \{0, 1\}^d$. Let a normal form be
\begin{align}
	h(x) & = M_1 \wedge M_2 \wedge M_3\\
	\text{s.t. } M_i & = L_{i1} \vee L_{i2} \vee ... \vee L_{i n_i}
\end{align}
where $\wedge$ is a logical AND, and $\vee$ is logical OR.\\
\\
A result of this is there is no polynomial time algorithm that can properly learn 3-DNF. However, if you allow for improper learning, you can learn it in polynomial time.
\end{sidework}

\subsection{Hardness of Improper Learning}
We've shown hardness of proper learning, can we show where improper learning breaks down?
\begin{definition}
	[Intersection of $k$ hard-spaces] Let $x \in \{\pm 1\}^d$
	\begin{align}
		h(x) = \prod_{i=1}^k \mathds 1[\langle w_j, x\rangle > 0]
	\end{align}
\end{definition}
Under assumption of hardness. For all $\epsilon > 0$, the internsection of $d^\epsilon$ hard-spaces is hard to learning even improperly.\\
\\
\emph{Remark:} $h(x)$ can be rewritten a 2-hidden layer ReLU, with 2k+1 neurons. So it seems that teacher-student models will fail to converge (in polynomial time), even when the teacher is a small model. Wow!


\section{Kernel Methods}
Let the tuple $(\Theta, \langle \cdot, \cdot \rangle)$ (which is a Hilbert space) be called a \textbf{feature space}. We have \textbf{feature maps} $\Phi: \mathcal X \to \Theta$, and we now consider the class of functions
\begin{align}
	\mathcal F = \{f(x; \theta) = \langle \theta, \Phi(x)\rangle_{\Theta} \}
\end{align}
A couple remarks
\begin{itemize}
	\item $\langle f(\cdot; \theta) , f(\cdot, \theta')\rangle_{\mathcal F} = \langle \theta, \theta' \rangle_{\Theta}$
	\item $|| f(\cdot ; \theta )||^2_{\mathcal F} = ||\theta||_{\Theta}^2 $
\end{itemize}
\begin{sidework}
	\emph{Example [Linear Regression]:} Let $\mathcal X = \mathbb R^d$ and $\Theta = \mathbb R^d$. Your feature map is $\Phi(x) = x$, and this mean your norm is $||f(\cdot; \theta)||_{\mathcal F} = ||\theta||_2$\\
	\\
	\emph{Example [Polynomial Regression]:} Let $\mathcal X = \mathbb R$, $\Theta = \mathbb R^{k + \ell}$, $\Phi(x) = (1, x, ..., x^k)$. Your model becomes $f(x; \theta) = \sum_i x_i \theta^k$\\
	\\
	\emph{Example [Infinite-Width 2 Layer Neural Network]:} This one is a bit more involved. First you must define a probability space $(\Omega, \mu)$. \\
	
	Your feature space is $\Theta = L^2(\Omega, \mu) = \{a: \Omega \to \mathbb R : \int a(\omega)^2 \mu (d\omega) < \infty\}$. Your inner-product on this space is just the continuous $L_2$, $\langle a, b\rangle_{\Theta}  = \int a(\omega) b(\omega) \mu(d\omega)$. Your feature map becomes $\Phi : \mathcal X \to L^2(\Omega, \mu)$, $x \mapsto \{\omega \mapsto \sigma(\langle x, \omega\rangle\}$. Your function class is $\mathcal F = \{f(x; a) = \int a(\omega) \sigma(\langle x, \omega\rangle ) \mu(d\omega) : a \in L^2(\Omega, \mu)\}$. 
\end{sidework}
\emph{Remark:} $(\mathcal F, \langle \cdot, \cdot \rangle_{\mathcal F})$ is a reproducing kernel Hilbert space (RKHS).
\begin{definition}
	[Reproducing Kernel Hilbert Space] Consider a function $K : \mathcal X \times \mathcal X \to \mathbb R$, which is symmetric in its argument, and positive semidefinite (where you imagine it's a infinite dimensional matrix, where indices are just the arguments).
	\begin{align}
		\forall x,f, ~ K(x, \cdot) \in \mathcal F \text{ s.t. } \langle f, K(x, \cdot) \rangle_{\mathcal F} = f(x)
	\end{align}
\end{definition}
\begin{theorem}
	[Moore-Arorgogin] For every symmetric, positive-definite kernel, there exists a unique RKHS.
\end{theorem}
\begin{sidework}
	\emph{Example:} Consider the Feature Space $(\Theta, \Phi)$, we can construct a RKHS by taking the the kernel to be $K(x,x') = \langle \Phi(x), \Phi(x') \rangle_{\Theta}$.\\
	\\
	In linear regression
	\begin{align}
		K(x,x') = \langle x, x'\rangle\\
	\end{align}
	In polynomial regression
	\begin{align}
		K(x,x') = \sum_{s=0}^k (x x')^s
	\end{align}
	In Infinite Width 2-layer NN
	\begin{align}
		K(x,x') = \int \sigma(\langle x, \omega \rangle ) \sigma(\langle x', \omega\rangle ) \mu(d\omega)
	\end{align}
\end{sidework}
\subsection{Kernel Method}
Let's define a problem to solve. Consider a neural network $f(\cdot, \theta) = \langle \hat \theta, \Phi(\cdot)\rangle_{\Theta}$ 
\begin{align}
	\hat \theta = \argmin_{ \theta \in \Theta} \left \{ \frac{1}{n} \sum_{i=1}^n \ell(y_i, \langle \theta, \Phi(x_i)\rangle_{\Theta} ) + \lambda ||\theta||^2_{\Theta} \right\}
\end{align}
\begin{align}
	\hat f = \argmin_{f \in \mathcal F} \left \{ \frac{1}{n} \sum_{i=1}^n \ell(y_i, f(x_i)) + \lambda ||f||_{\mathcal F}^2  \right \}
\end{align}

\begin{theorem}
	[Representor Theorem] Any solution $\hat \theta \in \text{span} (\Phi(x_1), ..., \Phi(x_n)\} \subseteq \Theta$\\
	\\
	\emph{Proof:} Let $\Theta = V_{||} \oplus V_{\perp}$ where $V_{\perp} = \{\theta \in \Theta: \langle \theta, \Phi(x_i) = 0~~ \forall i \in [n] \}$ (it's perpendicular w.r.t. the inner product defined on $\Theta$). You can decompose 
\begin{align}
	||\theta||^2_{\Theta} = ||\theta_{||}||^2_{\Theta} +||\theta_{\perp}||^2_{\Theta}
\end{align}
Now compute
\begin{align}
	\min_{\theta = \theta_{||} + \theta_{\perp}} \{ \frac{1}{n} \sum_{i=1}^n \ell(y_i , \langle \theta_{||}, \Phi(x_i)\rangle + \underbrace{\langle \theta_{\perp}, \Phi(x_i)\rangle}_{\to 0, \text{ by def of } \ theta_{\perp}} ) + \lambda ||\theta_{||}||^2 + \lambda \underbrace{||\theta_{\perp}||^2 }_{\to 0}
\end{align}
QED?
\end{theorem}
\subsection{Kernel Trick}
Consider $\theta = \sum_i a_i \Phi(x_i)$, which yields a model $f(x; \theta) = \sum_{i=1}^n a_ K(x, x_i)$. where $k_m(x) = (K(x, x_1), ..., K(x, x_n))$. The  minimization
\begin{align}
	\hat a = \argmin_{a \in \mathbb R^n} \{ \frac{1}{n} \ell(y_i, a^T K_n(x_i) + \lambda a^T K_n a\}
\end{align}
This is a sum of convex functions in $a$, so this is simple.
\begin{align}
	f(x; \hat a) = \sum a_i K(x, x_i)
\end{align}
This process is known as the \textbf{kernel trick}.

\subsection{Domions Lower Bound}
\begin{theorem}
	Let $(\mathcal R, \langle \cdot, \cdot \rangle_{\mathcal R})$ be a Hilbert Space. Let $T \subset \mathcal R$ and $\text{dim}(T) = m$. Let $\mathcal H = \{h_1,..., h_M\} \subseteq \mathcal R$.
	
	The average approximation error
	\begin{align}
		\epsilon \equiv \frac{1}{M} \sum_{i=1}^M \inf_{f\in T} ||f - h_i||^2_{\mathcal R}
	\end{align}
	Then you can show the bound
	\begin{align}
		m \geq \frac{M}{||G||_{op}} (1-\epsilon)
	\end{align}
	where $g = (\langle h_i, h_j\rangle)_{i,j \in [n]}$
\end{theorem}
Now let's apply this lower bound.

\begin{sidework}
	\emph{Example:} Consider you get data $x \sim p_{\mathcal X}$, and $h(x) = \mathbb E[y | x]$. We construct a dataset $(y_i, x_i)_{i \leq n}$.
	
	You have a set of basis function $h \in \mathcal H \in \{h_1,..., h_M\}$. You now consider a subspace $T = \text{span} \{ K(\cdot, x_i) : i \in [m] \}$
	
	\begin{align}
		\hat f_j \to || \hat f_j - h_j ||^2 _{L^2} \geq \inf_{f \in T} || f - h_j ||^2_{L^2}
	\end{align}
	You have 
	\begin{align}
		AAE \geq \frac{1}{M} \sum_j \inf_{f\in T} ||f - h_j||^2_{C^2} = \epsilon
	\end{align}
	To prove ${AAE} \leq \epsilon$, you must have number of samples $n\geq \frac{M}{||G|_{op}}(1-\epsilon)$, 
\end{sidework}

\section{Noisy Gradient Descent}
Consider a model $f(\cdot; \theta): \mathcal X \to \mathbb R$ s.t. $\theta \in \mathbb R^p$. Assume this model is differentiable, that is $\nabla_\theta f(x; \theta)$ exists for all $x,\theta$. Our loss is a mean-squared error $\ell(y , \hat Y) = (y - \hat y)^2$, meaning the population risk will be $\mathcal R(\theta) = \mathbb E_{\mathcal D}[(y - f(x;\theta))^2]$. We have diagnostics regression fit $h(x) = \mathbb E_{\mathcal D}[y  |x]$ and an excess risk $\mathcal R_h (\theta) = ||f(\cdot;\theta) - h||^2_{L^2}$.

Today, we'll consider minimizing the population risk using \textbf{online SGD}, this is the following algorithm
\begin{itemize}
	\item Initialize the weights from some distribution $\theta^0 \sim p_0$
	\item At each step $k$
	\begin{itemize}
		\item Sample data $(y_k, x_k)\sim \mathcal D$
		\item Do a weight update according to the gradient 
		\begin{align}
			\theta^{k+1} & = \theta^{k} - \eta_k \nabla_\theta \ell(y_k, f(x_k, \theta^k))\\
			& = \theta^k + \eta_k \underbrace{(y_k - f(x_k ; \theta^k)) \prod_{B(O,R)} \nabla_\theta f(x_k; \theta^k)}_{\equiv - \Delta (\theta^k; y^k , x^k)}\\
			& = \theta^k - \eta_k \underbrace{g(\theta^k)}_{\text{drift}} - \eta_k \underbrace{(\Delta(\theta^k; y^k, x^k) - g(\theta^k))}_{\text{noise / Martingale}}
		\end{align}
		where $g(\theta^k) = \mathbb E_{(y_k, x_k) \sim \mathcal D}[\Delta(\theta^k ; y_k, x_k)]$
	\end{itemize} 
\end{itemize}
Remarks
\begin{itemize}
	\item $z = \frac{1}{\sqrt{\eta}}$
	\item Notation: $g_h(\theta^k) \equiv g(\theta^k)$
\end{itemize}
You also have \textbf{noisy gradient descent}, this is a slightly different algorithm
\begin{enumerate}
	\item Sample an initial weight configuration $\theta^0 \sim p_0$
	\item Perform a gradient update $\theta_h^{k+1} = \theta_h^{k} - \eta_k g_h(\theta^k) + \eta_k \xi_k$, where $\xi_k \sim \mathcal N(0, \sigma^2 \mathbb I_p)$
\end{enumerate}
Our objective will be to find a lower-bound on $R_h(\theta_h^k)$.\\
\\
Hypothesis class $\mathcal H\subseteq \{h : \mathcal X \to \mathbb R\}$
\begin{definition}
	[Correlation Alignment Complexity (CAC)] Let $\mathcal H\subseteq \{h : \mathcal X \to \mathbb R\}$ by a hypothesis class that you'll test functions in, and $\mu_H$ be a prior over said hypothesis class. Also $\nu \in L^2(\mathbb P_{\mathcal X})$.
	\begin{align}
		\text{Align}_C(\mu_H, \nu) = \left\{\sup_{||\phi||_{L_2} \leq 1}\mathbb E_{h \sim \mu_H}[\langle \phi, h-\nu\rangle_{L^2}^2] \right\}
	\end{align}
	Theo created this measurement, how.
\end{definition}
Remark: Large CAC fits in $\mathcal H$ centered by reference $g$ are mainly orthogonal to any fixed direction.

\begin{theorem}
	[Avbbe, Boirn-Adsen, 2022]
	Consider $\theta_h^k$, which is the parameters after $k$ steps of noisy GD.
	\begin{align}
		\mathcal R_h (\theta_h^k) \geq ||h_k - v||_{L^2}^2  - \epsilon
	\end{align}
	with probability at least
	\begin{align}
		1 - \left ( \frac{R}{2z} \sqrt{\frac{k}{A}} + \frac{1}{\epsilon A} \right) 
	\end{align}
\end{theorem}
Remark
\begin{itemize}
	\item If $k$ is too small, we don't improve on the trivial prediction, i.e. you'll always output $\nu$. This means you want
	\begin{align}
		k \geqslant \left (\frac{z}{R}\right)^2 \text{Align}
	\end{align}
	\item $z/R$ is the gradient precision. If Align is large, we'll either need a large amount of steps or small enough precision. Meaning you want 
	\begin{align}
		fk \geqslant \frac{\text{Align}}{R^2}
	\end{align}
\end{itemize}
\subsubsection{Proof}
The idea is to use "junk flow". This means you compare a noisy GD trajectory to another noisy GD trajectory, where the label is replaced with a "junk" label. Basically, if you can show that the two trajectories stay close together, then you can show the original trajectory can't minimize the test error.

Consider noisy GD on true target $h$, $\{\theta_h^t\}_{t=0}^k$, and noisy GD on junk target $\nu$ $\{\theta_\nu^t\}_{t=0}^k$.

We can further break down this proof by constructing two lemmas (which we'll not prove)
\begin{lemma}
	\begin{align}
		\mathbb E_{h \sim \mu_H}[TV(\theta_nu^k, \theta_h^k)] \leq \frac{R}{2z} \sqrt{\frac{k}{A}}
 	\end{align}
 	There exists coupling $\theta_\nu^k, \theta_h^k$ s.t. $\mathbb P(\theta_h^k \neq \theta_\nu^k) \leq \frac{R}{2z} \sqrt{\frac{k}{A}}$
\end{lemma}
\begin{lemma}
	\begin{align}
		\mathbb P(R_h(\theta_\nu^k) \leq ||h - \nu||_{L^2}^2 - \epsilon) \leq \frac{1}{\epsilon A}
	\end{align}
\end{lemma}
Proof of theorem
\begin{align}
	\mathbb P(R_h(\theta_h^k)\leq t) & \leq \mathbb P(\theta_h^k \neq \theta_\nu^k) + \mathbb P(R_k(\theta^k_\nu) \leq t)\\
	= \frac{R}{2z} \sqrt{\frac{k}{A}} + \frac{1}{\epsilon A}
\end{align}
where we plug and chug the previous two lemmas



\section{Statistical Query Algorithms}
Consider a map $\phi: \mathcal Y \times \mathcal X \to [-1, 1]$, where you have an objective $\mathbb E_{\mathcal D} [\phi(y,x)]$. This framework is nice because it makes it easy to find tight lower bounds in the form, can the algorithm succseed with tolerance $\mathbb E_{\mathcal D}[\phi(y,x)] + \epsilon$ (for $\epsilon \in [- \delta, \delta]$).

You then ask 


















