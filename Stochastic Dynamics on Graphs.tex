%%% Document Formatting
\documentclass[12pt,fleqn]{article}
\usepackage[a4paper,
            bindingoffset=0.2in,
            left=0.75in,
            right=0.75in,
            top=0.8in,
            bottom=0.8in,
            footskip=.25in]{geometry}
\setlength\parindent{10pt} % No indent

%%% Imports
% Mathematics
\usepackage{amsmath} % Math formatting
\numberwithin{equation}{section} % Number equation per section
\DeclareMathOperator{\Tr}{Tr}


\newtheorem{theorem}{Theorem}
\newtheorem{definition}{Definition}
\newtheorem{proof}{Proof}
\newtheorem{lemma}{Lemma}
\newtheorem{example}{Example}

\usepackage{amsmath}
\usepackage{amsfonts} % Math fonts
\usepackage{amssymb} % Math symbols
\usepackage{mathtools} % Math etc.
\usepackage{slashed} % Dirac slash notation
\usepackage{cancel} % Cancels to zero
\usepackage{empheq}
\usepackage{breqn}

\newcommand{\expect}[1]{\mathbb{E}\left[#1\right]}


% Visualization
\usepackage{graphicx} % for including images
\graphicspath{ {} } % Path to graphics folder
\usepackage{tikz}



%%% Formating
\usepackage{hyperref} % Hyperlinks
\hypersetup{
    colorlinks=true,
    linkcolor=blue,
    filecolor=magenta,      
    urlcolor=cyan,
    pdftitle={Overleaf Example},
    pdfpagemode=FullScreen,
    }
\urlstyle{same}

\usepackage{mdframed} % Framed Enviroments
\newmdenv[ 
  topline=false,
  bottomline=false,
  skipabove=\topsep,
  skipbelow=\topsep
]{sidework} %% Side-work

\usepackage{lipsum} % Lorem Ipsum example text

%%%%% ------------------ %%%%%
%%% Title
\title{Mean Field Dynamics on Graphs}
\author{Clark Miyamoto (cm6627@nyu.edu)}
\date{\today}
\begin{document}

\maketitle
\section{Mean Field Dynamics}
Consider a mean field dynamics
\begin{align}
	dX_i^{(n)}(t) = b(t, X_i^n, \mu^n) dt + dB_i(t), ~~ i = 1,...,n
\end{align}
where $\mu^n = \frac{1}{n} \sum_{i=1}^n \delta_{X_i^{n}} \in \mathcal P(\mathcal C))$ (it's a sum of delta functions which evaluate to the current position of particle $i$). Since mean field is in the title, you can imagine you're taking $n \to \infty$. So the question becomes, what is $\lim_{n \to \infty}  dX_i^{(n)}(t)$ and $\lim_{n \to \infty} \mu^n$.\\
\\
What you can show is that
\begin{align}
	(X_1^{(n)}, ..., X_k^{(n)}) \to^{(d)} (Y, ..., Y)
\end{align}
where $Y$ is defined by
\begin{align}
	dY(t) = b(t,Y, \mu) dt + d\tilde B(t)
\end{align}
where $\mu = \text{Law}(Y)$. Think about how weird this is, your trajectories depend on your distribution...\\
\\
What you can show is
\begin{align}
	\lim_{n \to \infty }\mathbb E[ f(X_1^n ) g(X_2^n)]  = \mathbb E[f(Y) ] \mathbb E[g(Y)]
\end{align}
So there's some sort of self-averaging going on.

\section{Mean Field Dynamics on a Graph}
\subsection{Setup}
So imagine your particles interaction dependent on some graph structure $G = ([n], E^n)$. Notation: $\partial_{\nu}$ means the neighbors of particle $\nu$ endowed by the graph structure.
\begin{align}
	dX_\nu^{n}(t) = b(t, X_\nu, X_{\partial_\nu}) dt + dB_\nu (t), ~~ \nu \in G
\end{align}
basically your dynamics depend on yourself and your neighbors. \\
\\
\textbf{Goal:} Can you establish a similar framework here, that is make sense of the infinite particle? Can we see if $\frac{1}{n} \sum_{\nu \in G} \delta_{X_\nu^n} \to \mu$? And can make a statement on $X_i^n \to ?$ ?
\subsubsection{Graph Structure in Infinite Node Limit}
He went through a lot of pain to do this. I am not a mathematician, so I don't care. But basically, it's well defined.

\subsubsection{Dynamics in Infinite Node Limit}
Consider some graph $G = ([n], E^n)$, and some dynamics
\begin{align}
	dX_\nu(t) = b(t, X_\nu, X_{\partial_\nu}) dt + dB_\nu(t)
\end{align}
We find two different limits (due to the ways your take the graph to a high-node limit, i.e. do you average over the root, or do you average over the disorder in the graph topology, or both?)
\begin{enumerate}
	\item $G^n \to G$ locally weakly in distribution, then $X^{G^n} \to X$ locally weakly in distribution.
	\item $G^n \to G$ in probability, then $X^{G^n} \to X$ in probability. However this one is nicer, because $\frac{1}{n} \sum \delta_{C_{\nu}(X^n)} \to^{\mathbb P} \text{Law}(X^G)$. So we'll only work with this one because your mean field approximation is well defined here. This one you also get the self-averaging as well
	\begin{align}
		\lim_{n\to \infty} \mathbb E[f(G^n, \circ^n_1) g(G^n, \circ^n_2)] = \mathbb E[f(G, \emptyset )] \mathbb E[g(G, \emptyset)]
	\end{align}
	where $G$ is the graph, and $\emptyset$ is the root of the graph.
\end{enumerate}
\subsection{Example: A line}
Consider a graph which is just a straight line
\begin{align}
	...-0-\underbrace{0-0-0}_{\text{Here}}-0-...
\end{align}
Consider looking at the time evolution 
\begin{align}
	dX_\nu(t) = b(t, X_\nu, X_{\partial_\nu}) dt + dB_\nu(t)
\end{align}
Let's look at the root
\begin{align}
	dX_{\emptyset}(t) & = b(t,  X_{\emptyset},  X_{\{\pm 1\}}) dt + dB_{\emptyset}(t)\\
	& = \mathbb E[b(t,  X_{\emptyset}, X_{\{\pm 1\}}) | X_{\{-1, 0, 1\}}] dt + dB(t)
\end{align}

















































\end{document}



